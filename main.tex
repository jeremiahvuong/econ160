\documentclass{article}

\usepackage[fancysections,tcbtheorems,enumerate,bibtex]{preamble}

\title{Game Theory (Stanford)}
\author{Jeremiah Vuong}
\date{2025-05-12}

\begin{document}

% \pagestyle{fancy}
% \fancyhead[L]{Game Theory}
% \fancyhead[R]{\thepage}

\maketitle

\tableofcontents
\newpage

\lecture[]{Introduction to Game Theory}
Game Theory is a way of thinking about strategic interactions between self-interested people.
\begin{definition}[TCP Protocol (rough example)]
  You don't actually have a direct connection to the other person.
  Instead, there is a whole sequence of computers that pass your message along until it reaches the other person.
  After the message is sent, the other person will send an acknowledgement back to you (likewise through the same sequence of computers).

  There could be a situation where along the chain of computers, the message gets lost.
  Such that your computer handles this by waiting a certain amount of time and if it doesn't receive an acknowledgement,
  it assumes the message was never recieved.

  In such case, your computer slows down the rate at which it sends messages on the assumption that
  there is a congestion on the network.
\end{definition}

\begin{example}[TCP Protocol Game]
  Should you send your packets using the correctly implemented TCP protocol or should you send them without the protocol?

  This is what we call a two-player game:
  \begin{itemize}
    \item \textbf{both use protocol}: both get 1ms delay
    \item \textbf{one uses protocol, the other doesn't}: 4ms for correct, 0ms for defecting
    \item \textbf{both defect}: both get 3 ms delay
  \end{itemize}
\end{example}

Of course, assuming all players are rational, they would want to minimize their delay.
Such that risking the scenario where one player uses the protocol and the other doesn't is not optimal.

In a case where you could see what the other player is doing, you could use a strategy like tit-for-tat.

\subsection{Self-interested Agents and Utility Theory}
\begin{definition}[Self-interested Agents and Utility Theory]
  The agent has its own description of states of the world that it likes,
  and acts based on that description.

  Utility function:
  \begin{itemize}
    \item quantifies degree of preference across alternatives
    \item explains the impact of uncertainty
    \item \textbf{Decision-theoretic}: act to maximize expecetex utility
  \end{itemize}
\end{definition}
Ultimately, we can say that agents are perfectly rational if they act to maximize their utility.
And that the utility function is a way (dimensionally?) of representing their preferences.

\subsection{Defining Games}
\begin{definition}[Game]
  A game is a collection of players, actions, and payoffs. Where the key ingredients:

  \begin{itemize}
    \item Players: who are the decision-makers?
    \qitem{People, firms, companies, etc.}
    \item Actions: what can each player do?
    \qitem{Enter a bid, decide when to end an event, decide when to sell stock, etc.}
    \item Payoffs: what motivates players?
    \qitem{Profit, other players' actions, etc.}
  \end{itemize}
\end{definition}

Note that a "game" is simply a singular scenario, instead of a collection of scenarios.
There are two standard representations of games:

\begin{definition}[Normal Form (aka Matrix form, Strategic Form)]
  List what players get as a function of their actions. 
  It is as if players choose their actions simultaneously.

  For a finite, $n$-person game: $\langle N, A, u \rangle$
  \begin{itemize}
    \item \textbf{Players:} $N = \{1, \ldots, n\}$ is a finite set of $n$ players, indexed by $i$
    \item \textbf{Action set:} for a player $i$,
    $$a = (a_i, \ldots, a_n) \in A = A_i \times \ldots \times A_n$$
    Where $A$ is actions chosen by all players (action profile),
    $A_i$ is the set of actions available to player $i$,
    $a$ is an ordered tuple of what each player is doing,
    and $a_i$ is a representative action for player $i$.
    \item \textbf{Utility function (payoff):} for player $i$: $u_i: A \to \mathbb{R}$
    \qitem{$u = (u_1, \ldots, u_n)$ is a profile of utility functions.}
  \end{itemize}
\end{definition}


\begin{example}[TCP Backoff Game in Matrix Form (Normal)]
  Two players are representitive as either a row or column player.
  The first number is for the row player.
  Where delay (utility) is measured in milliseconds and negative.
  \begin{center}
    \begin{tabular}{|c|c|c|}
      \hline
      & \textbf{Use} & \textbf{Don't} \\
      \hline
      \textbf{Use} & (-1, -1) & (-4, 0) \\  
      \textbf{Don't} & (0, -4) & (-3, -3) \\
      \hline
    \end{tabular}
  \end{center}
\end{example}

\newpage

\begin{example}[Large Collective Game]
  Let's assume we have a game where there are too many players
  to matrix out all the utilities.
  The players can choose to either revolt or not against their leader.
  \begin{itemize}
    \item \textbf{Players:} $N = \{ 1, \ldots, 10,000,000\}$
    \item \textbf{Actions:} $A_i = \{ \text{Revolt}, \text{Not Revolt}\}$
    \item \textbf{Utility} for player $i$:
    \begin{itemize}[label=$\bullet$]
      \item $u_i(a) = 1$ if $\#\{ j : a_j = \text{Revolt}\} \geq 2,000,000$
      \item $u_i(a) = -1$ if $\#\{ j : a_j = \text{Revolt}\} < 2,000,000$ and $a_i = \text{Revolt}$
      \item $u_i(a) = 0$ if $\#\{ j : a_j = \text{Revolt}\} < 2,000,000$ and $a_i = \text{Not Revolt}$
    \end{itemize}
    Where 1 is a successful revolt, -1 is a failed revolt (punishment), and 0 is not revolting.
  \end{itemize}
\end{example}
In games where the outcome is ultimately decided by the players,
in cases where we can orginize, it is beneficial to do so.
In the Large Collective Game, the revolt is successful if at least 2,000,000 players revolt,
even if you yourself do not revolt.
In the case that not enough players revolt, it is on you to decide your payoff.
If you were a player that didn't care about the outcome (revolt), you would not revolt, as too keep yourself from punishment.

\begin{definition}[Extensive Form]
  Includes the timing of moves.
  \begin{itemize}
    \item Players move sequentially, represented as a tree.
    \qitem{Chess is a good example of this, where white and black move alternately.}
    \item Keeps track of what each player knows at each point in the game.
    \qitem{Poker: sees bet not necessarily what cards they have.}
  \end{itemize}
\end{definition}




\end{document}

